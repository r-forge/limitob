\documentclass[a4paper]{report}
\usepackage[round]{natbib}


\usepackage{Rnews}
\usepackage{fancyvrb}
\usepackage{Sweave}

\DefineVerbatimEnvironment{Sinput}{Verbatim}{fontsize=\small,fontshape=sl}
\DefineVerbatimEnvironment{Soutput}{Verbatim}{fontsize=\small}
\DefineVerbatimEnvironment{Scode}{Verbatim}{fontsize=\small,fontshape=sl}

%% \SweaveOpts{prefix.string=graphics/portfolio}

\bibliographystyle{abbrvnat}

\begin{document}
\begin{article}

\title{orderbook}
\author{David Kane, Andrew Liu and Khanh Nguyen}

%%\VignetteIndexEntry{Using the orderbook package}
%%\VignetteDepends{orderbook}

\maketitle


\setkeys{Gin}{width=0.95\textwidth}

\section*{Introduction}

The \pkg{orderbook} package provides facilities for exploring and
visualizing the data associated with an order book: the electronic
collection of the outstanding limit orders for a financial instrument,
e.g. a stock. A \emph{limit order} is an order to buy or sell a given
quantity of stock at a specified limit price or better. The
\emph{size} is the number of shares to be bought or sold.  An order
remains in the order book until fully executed, i.e. until its size is
zero as a result of trades. Partial executions occur as a result of
trades for less than the entire size of the order.

Consider a simple order book containing five limit orders: sell 150
shares of IBM at \$11.11, sell 150 shares of IBM at \$11.08, buy 100
shares of IBM at \$11.05, buy 200 shares of IBM at \$11.05, and buy
200 shares of IBM at \$11.01.

\begin{verbatim}
                 Price          Ask Size

                 $11.11         150
                 $11.08         100
     300         $11.05
     200         $11.01

Bid Size         Price
\end{verbatim}

\noindent Orders on the \emph{bid} (\emph{ask}) side represent orders to buy
(sell). The price levels are \$11.11, \$11.08, \$11.05, and
\$11.01. The \emph{best bid} at \$11.05 (highest bid price) and the
\emph{best ask} at \$11.08 (lowest ask price) make up the \emph{inside
  market}. The \emph{spread} (\$0.03) is the difference between the
best bid and best ask. The \emph{midpoint} (\$11.065) is the average
of the best bid and best ask.

There are four types of messages that traders can submit to an order
book: \emph{add}, \emph{cancel}, \emph{cancel/replace}, and
\emph{market order}. A trader can \emph{add} a limit order into the
order book.  She can also \emph{cancel} an order and remove it from
the order book. If a trader wants to reduce the size of her order, she
can issue a \emph{cancel/replace}, which cancels the order, then
immediately replaces it with another order at the same price, but with
a lower size. Every limit order is assigned a unique ID so that cancel
and cancel/replace orders can identify the corresponding limit
order. A \emph{market order} is an order to immediately buy or sell a
quantity of stock at the best available prices. A trade occurs when a
market order is paired with the correspnding limit order in the inside market.

All orders have timestamps indicating the time at which they were
accepted into the order book. The timestamp determines the \emph{time
  priority} of an order. Earlier orders are executed before later
orders. For example, suppose that the order to buy 100 shares at
\$11.05 was submitted before the order to buy 200 shares at
\$11.05. Now suppose a market order selling 200 shares is submitted to
the order book. The limit order for 100 shares will be executed
because it is at the front of the queue at the best bid. Then, 100
shares of the order with 200 total shares will be executed, since it
was second in the queue. 100 shares of the 200 share order remain in
the order book at that price level.

A market order for more shares than the size at the inside market will
execute at worse price levels until it is complete. For example, if a
market order to buy 200 shares is submitted to the order book, the
order at \$11.08 will be fully executed. Since there are no more
shares available at that price level, 100 shares at the \$11.11 price
level will be transacted to complete the market order. An order to
sell 50 shares at \$11.11 will remain in the order book. Executing
these two market orders on our hypothetical order book results in a
new state of the order book.

\begin{verbatim}
                 Price          Ask Size

                 $11.11         50
     100         $11.05
     200         $11.01

Bid Size         Price
\end{verbatim}

Note that cancel/replace orders can lower the size of an
order, but not increase it. Cancel/replace orders maintain the time
priority of the original order, so if size increases were allowed,
traders with orders at the highest time priority for a price level
could perpetually increase the size of their order, preventing others
from being able to buy or sell stock using limit orders at that price
level.

See \cite{johnson:barry} for more details on the order book.

\section*{Examples}

NVIDIA is a graphics processing unit and chipset developer with ticker
symbol NVDA. Consider the order book for NVDA at a leading electronic
exchange on June 8, 2010. We create the \texttt{orderbook} object by
specifying the location of our data file.

\begin{Schunk}
\begin{Sinput}
> library(orderbook)
> filename <- system.file("data", "sample.txt", package = "orderbook")
> ob <- orderbook(file = filename)
> ob <- read.orders(ob, 10000)
\end{Sinput}
\end{Schunk}
\begin{Schunk}
\begin{Sinput}
> ob
\end{Sinput}
\begin{Soutput}
An object of class orderbook (default)
--------------------------
Current orderbook time:    09:35:02 
Message Index:             10,000 
Bid Orders:                631 
Ask Orders:                1,856 
Total Orders:              2,487 
\end{Soutput}
\end{Schunk}

We read in the first 10,000 messages then \texttt{show} the object.
The current time is 9:35:02 AM. The message index indicates which row
in the data file the object has read through. The display also shows
that there are 631 bids and 1,856 asks outstanding, for a total of
2,487 orders. This indicates that many earlier orders have been
removed through either cancels or trades.

\begin{Schunk}
\begin{Sinput}
> summary(ob)
\end{Sinput}
\begin{Soutput}
Current time is 09:35:02 

Ask price levels:   540 
Bid price levels:   179 
Total price levels: 719 
-----------------------------
Ask orders:         1,856 
Bid orders:         631 
Total orders:       2,487 
-----------------------------
Spread:             0.02 

Mid point:          11.37 
-----------------------------
Inside market 
 
Best Bid:           11.36 
Size:               2,700 
 
Best Ask:           11.38 
Size:               400 
\end{Soutput}
\end{Schunk}

Using \texttt{summary} the total order information from \texttt{show}
is repeated. Additionally, we see that there are 540 ask and 179 bid
price levels, for a total of 719. This indicates that many orders
have been submitted at the same price level.  The spread is \$0.02,
and the midpoint is \$11.37. The inside market is composed of 2,700
shares offered at the best bid of \$11.36 and 400 shares offered at
the best ask of \$11.38.

\begin{Schunk}
\begin{Sinput}
> display(ob)
\end{Sinput}
\begin{Soutput}
Current time is 09:35:02 

		 Price 	 Ask Size
---------------------------------------------
		 11.42 	 900 
		 11.41 	 1,400 
		 11.40 	 1,205 
		 11.39 	 1,600 
		 11.38 	 400 
---------------------------------------------
  2,700 	 11.36 
  1,100 	 11.35 
  1,100 	 11.34 
  1,600 	 11.33 
    700 	 11.32 
---------------------------------------------
Bid Size 	 Price
\end{Soutput}
\end{Schunk}

\texttt{display} shows the inside market, along with the four next
best bid and ask price levels and the size at each price level.

\begin{figure}
\centering
\vspace*{.1in}
\begin{Schunk}
\begin{Sinput}
> plot(ob)
\end{Sinput}
\end{Schunk}
